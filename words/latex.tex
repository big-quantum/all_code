\documentclass{article}

\usepackage{ctex}
\usepackage{xltxtra}
\usepackage{texnames}
\usepackage{mflogo}
\usepackage{amsmath}
\usepackage{amssymb}



\title{LateX特殊字符}
\author{WSKH}
\date{\today}

\begin{document}
	\maketitle
	\section{空白符号}
	空白格 空白格
	
	空白格        空白格
	
	Hello Hello
	
	Hello      Hello
	\subsection{空行分段,多个空行等同于1个}
	\subsection{自动缩进,绝对不能使用空格代替}
	\subsection{英文中多个空格处理为1个空格,中文中空格将被忽略}
	\subsection{汉字与其他字符的间距会自动由XeLateX处理}
	\subsection{禁止使用中文全角空格}
	
	\section{控制字体宽度}
	\subsection{控制空格}
	a\quad b\par              %1em(当前字体中M的宽度)
	a\qquad b\par			  %2em
	a\,b a\thinspace b\par    %约为1/6em
	a\enspace b\par			  %0.5em
	a\ \ \ \ b\par            %4个空格
	a~b\par                   %硬空格(不能分割的空格)
	
	\subsection{根据参数产生指定间隔}
	a\kern 1pc b\par
	a\kern -1em b\par     %b距离a-1em,相当于把ab反向输出为ba
	a\hskip 1em b\par
	a\hspace{35pt}b\par
	
	\subsection{根据指定字符的宽度来产生空白}
	a\hphantom{1}b\par
	a\hphantom{12}b\par
	a\hphantom{123}b\par
	a\hphantom{1234}b\par
	a\hphantom{12345}b\par
	
	\subsection{弹性长度空白}
	a\hfill b\par   %弹性空白可以让两边的字符正好到达纸张边界
	
	\section{LateX特殊字符}
	\subsection{控制符}
	\#   \$   \%   \{\}   \~{}   \_{}   \^{}   \textbackslash   \&
	
	\subsection{排版符号}
	\S  \P  \dag  \ddag  \copyright  \pounds
	
	\subsection{LateX标志符号}
	\subsubsection{默认自带}
	\TeX{}   \LaTeX{}   \LaTeXe{}
	\subsubsection{xltxtra宏包提供}
	\XeLaTeX
	\subsubsection{texnames宏包提供}
	\AmSTeX{}    \AmS-\LaTeX{}
	\BibTeX{}    \LuaTeX{}
	\subsubsection{mflogo宏包提供}
	\METAFONT{}    \MF{}    \MP{}
	
	\subsection{引号}
	`   '   ``   ''    ``你好''    `好'
	
	\subsection{连字符}
	-    --    ---
	
	\subsection{非英文字符}
	\oe  -  \OE  -  \ae  -  \AE  -  \aa  -  \AA  -  \o  -  \O  -  
	\l   - \L  -  \ss  -  \SS  -  !`   -  ?`
	
	\subsection{重音符号(以o为例)}
	\`o - \'o - \^o - \''o - \~o - \=o - \.o - \u{o} - \v{o} - \H{o} - 
	\r{o} - \t{o} - \b{o} - \c{o} - \d{o}

  \section{分式}
	$\frac{1}{5}$

	
  \begin{tabular}{l|c|p{1.5cm}|c|r}  %l:左对齐 c:居中 r:右对齐 |:在该位置增加竖线 p{1.5cm}指定宽度
		\hline %\hline为表格增加一条横线
		姓名 & 语文 & 数学 & 外语 & 备注 \\ %\\结束这一行表格的书写
		\hline
		张三 & 98 & 96 & 97 & 优秀 \\
		\hline
		李四 & 85 & 87 & 83 & 良好 \\
		\hline
		王五 & 73 & 74 & 77 & 一般 \\
		\hline
		赵六 & 62 & 66 & 64 & 及格 \\
		\hline
	\end{tabular}

  \section{上下标}
	\subsection{上标}
	$x^2+y^4=z^3$
	\subsection{下标}
	$P_i+N_j=M_k$

  \section{希腊字母}
	$\alpha$
	$\beta$
	$\gamma$
	$\epsilon$
	$\pi$
	$\omega$
	
	$\Gamma$
	$\Delta$
	$\Theta$
	$\Pi$
	$\Omega$

  \section{数学函数}
	$\log$
	$\sin$
	$\cos$
	$\arcsin$

  \section{多行公式}
	% gather环境
	\subsection{gather带编号}
	\begin{gather}
		a+b=b+a \\
		f(x)=x^2+x+1
	\end{gather}
	\subsection{gather*不带编号}
	\begin{gather*}
		a+b=b+a \\
		f(x)=x^2+x+1
	\end{gather*}
	% align环境,用&进行对齐
	\subsection{align}
	\begin{align}
		a+b&=b+a \\
		f(x)&=x^2+x+1
	\end{align}
	\subsection{align*}
	\begin{align*}
		a+b&=b+a \\
		f(x)&=x^2+x+1
	\end{align*}
	% split环境,对其采用align环境对齐方式,编号在中间
	\subsection{split}
	\begin{equation}
		\begin{split}
			a+b&=b+a \\
			f(x)&=x^2+x+1
		\end{split}
	\end{equation}
	% cases环境,每行公式用&分割为两个部分,通常表示为值和后面的条件
	\subsection{cases}
	\begin{equation}
		f(x)=\begin{cases}
			1,&x>0 \\
			0,&x=0 \\
			-1,&x<0
		\end{cases}
	\end{equation}


  \section{矩阵}
	$\begin{matrix}
		0 & 1 \\
		1 & 0
	\end{matrix}$

	$\begin{pmatrix}
		0 & 1 \\
		1 & 0
	\end{pmatrix}$

	$\begin{bmatrix}
		0 & 1 \\
		1 & 0
	\end{bmatrix}$

	$\begin{Bmatrix}
	0 & 1 \\
	1 & 0
	\end{Bmatrix}$

	$\begin{vmatrix}
		0 & 1 \\
		1 & 0
	\end{vmatrix}$

	$\begin{Vmatrix}
	0 & 1 \\
	1 & 0
	\end{Vmatrix}$

\end{document}
